\documentclass[aspectratio=169]{beamer}
\usepackage{amsmath, amssymb, amsthm}
\usepackage{NUS/beamerthemeNUS}
\usetheme{NUS} % Or aspectratio=43 for 4:3

\renewcommand{\emph}[1]{{\color{NUSOrange}\textbf{#1}}}


\title{An Extremely Long Presentation Title That's Used for Testing Aesthetics on the Title Page}
\subtitle{ABC Conference\\Earth, Solar System, The Milky Way}
\author[Your Name]{Your Name\\
        In Collaboration with Other names\\
        Department, University}
\institute{Your Institute}
\date{\today}

\begin{document}

\begin{frame}
    \titlepage
\end{frame}

\begin{frame}
    \frametitle{Table of Contents}
    \tableofcontents
\end{frame}

\section{Introduction}
\begin{frame}
    \frametitle{Introduction}
    \framesubtitle{A Subtitle Which Is Usually Longer}
    
    \begin{columns}
        \begin{column}{0.5\textwidth}
            \linespread{1.5}\selectfont % Temporarily adjust line spacing for this frame only
            \begin{itemize}
                \item This is an introduction sentence which could be long
                \begin{itemize}
                    \item This is a nested item 1
                    \item This is a nested item 2
                \end{itemize}
                \item This is another introduction sentence. \emph{Emphasize} this word.
                \item This is another another sentence.
            \end{itemize}
        \end{column}
        \begin{column}{0.5\textwidth}
            \begin{figure}
                \includegraphics[width=0.6\linewidth]{assets/nus-logo-blue-horizontal.jpg}
                \caption{NUS Logo}
                \label{fig:logo}
            \end{figure}
        \end{column}
    \end{columns}
\end{frame}

\section{Main Content}
\begin{frame}
    % \linespread{1.2}
    \frametitle{Main Topic}
    \begin{block}{This is a block.}
        This is a block. Write some symbols $a,b,c,$. And
        \begin{equation}
            a=\dfrac{\sum_{a}^{b}c+d}{e+f}.
        \end{equation}
    \end{block}
    \begin{exampleblock}{This is an orange block.}
        This is an orange block. Write other symbols $\alpha, \beta$, and
        \begin{equation}
            \alpha=\beta\times\dfrac{\alpha}{\beta}.
        \end{equation}
    \end{exampleblock}
\end{frame}

\section{Conclusion}
\begin{frame}
    \frametitle{Conclusion}
    \begin{itemize}
        \item This is one bullet point. It can be a long sentence.
        \item This is another bullet point. It can be another long sentence.
        \item This is another another bullet point. It can be yet another long sentence.
    \end{itemize}
\end{frame}

\end{document}